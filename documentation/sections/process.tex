\subsection{Requirements}
The following requirements for the software were identified:
\begin{itemize}
    \item The software transforms CSV files containing standardized patient data according to §21 Paragraph 4 and 5 KHEntgG into an OMOP CDM.
    \item The software can be configured to perform the transformation periodically.
    \item The software provides a prediction based on patient data of the patients risk developing a pressure ulcer (decubitus).
    \item The prediction on the risk of decubitus development can be verified on provided reasons leading to the classification.
    \item The generated prediction can be displayed in a web browser.
    \item The access to the patient data via the web browser is protected by a password authentication.  
\end{itemize}

\subsection{Tests}
The sub-modules of the software are tested independently due to the fact they only interact via defined interfaces in form of database entries.
Therefore only the correctness from one interface to another is tested.

In addition, one complete run from CSV to front-end is performed to ensure the correct interaction via the interfaces.

\subsubsection{ETL-Process}
The different functions of the ETL-Process are tested by unit tests.
Further, the complete sub-module is tested by inserting a set of CSV files into the OMOP CDM and checking the database content to contain the expected data.

\subsubsection{Decubitus Risk Prediction}
The decubitus risk prediction is based on a neural network, therefore the prediction can only be tested on test data. 
The F1-Score, recall, and precision are used to numerically evaluate the results.
Further, the returned top ten clinical findings are evaluated via visual inspection of the results. 

\subsubsection{Front-End}
The front-end was tested visually by displaying known data and checking the visualization for correctness.


\subsection{Change Management}
Updates for the subsections of the software can be deployed independently.
Therefore the change management for each segment needs to be analyzed separately.
This enables the fast deployment of bug fixes because only the affected sub-module needs to be updated, tested and deployed. 

\subsubsection{ETL-Process}
If the ETL-Process is updated it will be validated on test data.
Afterward it can be deployed.
The deployment simply consists of the replacement of affected Python files.
This can easily be done by using a versioning system like git. 
If a versioning system is used, the deployment only consists of the provider pushing the newest version and the customer pulling the newest version.

However, it should be noted that data already entered in the OMOP CDM will not be updated retroactively.
Meaning, erroneously data in the database can only be fixed by deleting affected entries and inserting it as a set of CVS files into the updated ETL-Process.

\subsubsection{Decubitus Risk Prediction}
The update routine of the decubitus risk prediction can be split into two different cases.

In the first case, the update fixes a bug in the prediction pipeline. 
A bug is detected and fixed in the code, followed by testing of the sub-module.
Afterward the new version can be deployed similar to the ETL-Process.

In the second case, the neural network (NN) is updated.
Assuming the trainset is extended by new data the NN can be retrained on the new data.
If the new network achieves better test results than the old version it can be deployed.
For this purpose, the file that defines the NN is updated and deployed similar to the ETL-Process.


\subsubsection{Front-End}
If the front-end is updated, it will be validated through usability-tests\footnote{tests, where one or more developers validate the software through simply using it}, where the patient data is replaced with dummy data. 
If these tests succeed, the sub-module can be deployed by replacing the contained Python, CSS, JavaScript and HTML files.
As the front-end only requires reading access to the database, two instances of different versions can run separately. 
Hence, if a newer version contains bugs or missing functionality, one can fall back to the previous version and report the issue to the developers.
Therefore the front-end offers high availability. 



%The purpose of our product is to predict potential decubitus patients with aid of patient specific medical records. In order to achieve a system of this capability, many steps are required.
%We need to constantly convert data into the OMOP CDM, evaluate the data and illustrate the results on doctors or medical assistants, hence they are able to treat the patients correctly. 

%\subsection{Development of an ETL Process}
%The first step of the product is to convert our standardized data into the OMOP CDM. 
%In order to achieve that, we write a Python framework, reading the data of the input CSV files, connect to the required database and saves the converted data.

%\subsection{Evaluation of the data}
%To achieve the best possible results, we focus on a deep learning approach, hence our system learns from real data, how likely it is that a patient suffers on decubitus. 
%Secondarily we work on a pure statistically approach as base line.
%If the second approach ends up more reliable, we discard the deep learning approach.
%As training data, we use a set of medical records from dummy patients.
%Unfortunately we don't have the opportunity to train on real patient data, as OHDSI don't offer real data of patients who actually suffered from decubitus. 

%\subsection{Building the app}
%We build a simple web python application for the use of our medical device.
%This app is possible to schedule when the ETL process should restart, configure the database settings and illustrate the results of the patient evaluation. 
