\subsection{Requirements}

\begin{itemize}
	\item For confidential reasons, a password for using the app is required
	\item The ETL process converts the standardized CSV files correctly into the OMOP CDM
	\item The ETL process can be scheduled through a cron job
	\subitem The cron job can be modified over the frontend of the app
	\subitem A ETL process can be triggered out of schedule over the frontend
	\item The evaluation of potential decubitus patients is aimed to be as accurate as possible for us
	\item The results of the evaluation can be illustrated inside the frontend of the app
	\subitem Selected additional data of the patient is illustrated
	\subitem Additional information can be showed, if desired
\end{itemize}

\subsection{Tests}

\subsubsection{ETL-Process}

\begin{tabular}{|c|c|}
	\hline
	input & expected output \\
	\hline
	correct set of csv files as input & data is converted correctly \\
	missing CSV file & error message appears \\
	CSV file with wrong content & error message appears \\
	\hline
\end{tabular}

\subsubsection{Evaluation of Patients}

As we are using a deep learning approach, we split our data in two sets: the training data and the test data. As the names suggest, the training data is only for train our 
network, which will then be evaluated using the test set. 

\subsubsection{Frontend}

Frontend tests are difficult, if not impossible. Therefore we can only test it by letting certain people evaluating it. 

