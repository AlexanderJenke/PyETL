The purpose of our product is to predict potential decubitus patients with aid of patient specific medical records. In order to achieve a system of this capability, many 
steps are required. We need to constantly convert data into the OMOP CDM, evaluate the data and illustrate the results on doctors or medical assistants, hence they are able 
to treat the patients correctly. 

\subsection{Development of an ETL Process}
The first step of the product is to convert our standardized data into the OMOP CDM. In order to achieve that, we write a Python framework, reading 
the data of the input CSV files, connect to the required database and saves the converted data.

\subsection{Evaluation of the data}
To achieve the best possible results, we focus on a deep learning approach, hence our system learns from real data, how likely it is that a patient suffers on decubitus. Secondarily 
we work on a pure statistically approach as base line. If the second approach ends up more reliable, we discard the deep learning approach. As training data, we use a set 
of medical records from dummy patients. Unfortunately we don't have the opportunity to train on real patient data, as OHDSI don't offer real data of patients who actually suffered 
from decubitus. 

\subsection{Building the app}
We build a simple web python application for the use of our medical device. This app is possible to schedule when the ETL process should restart, configure the database settings and illustrate 
the results of the patient evaluation. 
 
